% !TEX root = main.tex
\section{Nowa implementacja modułu AthenaMonitoring}
Głównym celem refaktoringu framworka, było oddzielenie kodu odpowiedzialnego za zarządzanie histogramami od kodu algorytmów. 
Dzięki takiemu rozgraniczeniu odpowiedzialności, twórcy algorytmów mogą skupić się na ich poprawnej implementacji i określeniu które wartości powinny znaleźć się na wykresach jako wynik. 
Natomiast to kiedy stworzyć histogram, jak go wypełnić i zapisać jest obsługiwane przez wspólny kod. 

Sercem nowego rozwiązania jest GenericMonitoringTool - nowe narzędzie w ramach frameworka Athena.
Zajmuje się on przygotowaniem histogramów i powiązaniem ich z kodem uruchamianych algorytmów. 
Działa on w oparciu o deklaracje.
Twórca algorytmu deklaruje zbiór wartości (obejmuje to skalary oraz tablice) jakie mogą być monitorowane w obrębie jego wykonania, oraz informuje GenericMonitoringTool kiedy są one gotowe do przekazania do histogramów.
Natomiast użytkownik takiego algorytmu, deklaruje jakie wykresy chciałby stworzyć i z użyciem których zmiennych. 
Następuje to w kroku konfiguracyjnym Athena Python, gdzie definiuje on typy histogramów, parametry binowania, zakresy wartości, opisy itd. 
Taki interfejs pozwala uniknąć rekompilacji kodu C++ za każdym razem, gdy użytkownik zmienia parametry wykresu. 

\comment{
GenericMonitoringTool umożliwił centralne zarządzanie kodem odpowiedzialnym za komunikację z frameworkiem ROOT oraz pozwolił na łatwiejsze  

Pozwala to ograniczyć kopiowanie kodu w obrębie algorytmów 
Daje to duża dowolność 
}


\comment{
\begin{itemize}
\item jak używać -> interface
\item jakie można uzyskać efekty
\item TESTY WYDAJNOŚCI + wielowątkowe
\item czemu warto używać
\item co potrafi zrobić
\item jak skonfigurować
\item opis kodu
\end{itemize}
}