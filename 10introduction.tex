% !TEX root = main.tex
\section{Wstęp}
Celem niniejszej pracy jest opisanie narzędzia \mbox{GenericMonitoringTool} stworzonego w ramach środowiska Athena\cite{Athena} i opisanie jak umożliwiło ono monitorowanie jakości danych w wielowątkowych algorytmach. 
Monitorowanie danych odbywa się poprzez generację histogramów i porównanie ich z referencjami. Opisane w pracy narzędzie dotyczy pierwszego kroku w tym procesie.
Histogramy używane są z tego powodu że pozwalają na kompaktowe podsumowanie istotnych cech dużego zbioru danych. 
W zastosowaniu do monitorowania jakości danych pozwalają na szybkie uchwycenie anomalii co w konsekwencji prowadzi do wdrożenia poprawek i polepszenia jakości danych czy wręcz do zlokalizowania krytycznych błędów w konfiguracji eksperymentu. 

\subsection{Eksperyment ATLAS} \comment{https://atlas.cern/discover/about}
ATLAS~\cite{} jest jednym z czterech głównych eksperymentów w ramach Wielkiego Zderzacza Hadronów(LHC)~\cite{} w ośrodku naukowym CERN. Jest to eksperyment fizyki cząstek elementarnych ogólnego przeznaczenia prowadzony przez międzynarodową grupę naukowców. Został zaprojektowany w celu pełnego wykorzystania potencjału odkrywczego i ogromnego zakresu możliwości fizycznych, jakie zapewnia LHC.

\subsection{Athena} \comment{https://atlassoftwaredocs.web.cern.ch/athena/}
Athena jest platformą programistyczną, która umożliwia pracę z danymi pochodzącymi z eksperymentu ATLAS. Obejmuje to m.in. filtracji danych, generowanie przypadków zdarzeń metodami Monte-Carlo, symulacje odpowiedzi detektora oraz rekonstrukcję danych. Jego głównymi komponentami są algorytmy, które analizują i przekształcają dane pochodzące z eksperymentu. Pozostałe komponenty środowiska to narzędzia (tools), za pomocą których implementowany jest wzorzec strategia.  Komponenty te pisane są w języku C++ a konfigurowane z użyciem skryptów napisanych w języku Python. 

\subsection{ROOT} \comment{https://root.cern.ch/}
ROOT~\cite{} jest środowiskiem stworzonym w CERN i Fermilab, które umożliwia pracę z wielkimi zbiorami danych, w celu ich statycznej analizy, wizualizacji i przechowywania. Został on stworzony w języku C++ przez co dobrze komponuje się z frameworkiem Athena i jest w nim używany na potrzeby prezentowania wyników działania algorytmów. 

\comment{
\begin{itemize}
\item cel pracy -> istniejacy monitoring, ulepszenie
\item co zostalo zrobione, zeby osiagnac cel -> napisanie frameworku do monitoringu
\item dlaczego jest to wazne? -> latwiejsze uzycie kodu, utrzymanie, mniejse objectosciowo, przygotowanie do zrownoleglenia
\end{itemize}
}