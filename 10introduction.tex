% !TEX root = main.tex
\section{Wstęp}
Celem niniejszej pracy jest opisanie narzędzia \mbox{GenericMonitoringTool} stworzonego w ramach modułu \mbox{AthenaMonitoring} stanowiącego część środowiska Athena\cite{Athena} i opisanie jak umożliwiło ono monitorowanie jakości danych w wielowątkowych algorytmach. 
Monitorowanie danych odbywa się poprzez generację histogramów i porównanie ich z referencjami. Opisane w pracy narzędzie dotyczy pierwszego kroku w tym procesie.
Histogramy używane są z tego powodu że pozwalają na kompaktowe podsumowanie istotnych cech dużego zbioru danych. 
W zastosowaniu do monitorowania jakości danych pozwalają na szybkie uchwycenie anomalii co w konsekwencji prowadzi do wdrożenia poprawek i polepszenia jakości danych czy wręcz do zlokalizowania krytycznych błędów w konfiguracji eksperymentu. 

\subsection{Eksperyment ATLAS} \comment{https://atlas.cern/discover/about}
ATLAS~\cite{} jest jednym z czterech głównych eksperymentów w ramach Wielkiego Zderzacza Hadronów(LHC)~\cite{} w ośrodku naukowym CERN. 
Jest to eksperyment fizyki cząstek elementarnych ogólnego przeznaczenia prowadzony przez międzynarodową grupę naukowców. 
Został zaprojektowany w celu pełnego wykorzystania potencjału odkrywczego i ogromnego zakresu możliwości fizycznych, jakie zapewnia LHC.

\subsection{Athena} \comment{https://atlassoftwaredocs.web.cern.ch/athena/}
Athena jest platformą programistyczną, która umożliwia pracę z danymi pochodzącymi z eksperymentu ATLAS. 
Obejmuje to m.in. filtracji danych, generowanie przypadków zdarzeń metodami Monte-Carlo, symulacje odpowiedzi detektora oraz rekonstrukcję danych. 
Jego głównymi komponentami są algorytmy, które analizują i przekształcają dane pochodzące z eksperymentu. 
Pozostałe komponenty środowiska to narzędzia (tools), za pomocą których implementowany jest wzorzec strategia. 
Komponenty te pisane są w języku C++ a konfigurowane z użyciem skryptów napisanych w języku Python. 

\subsection{ROOT} \comment{https://root.cern.ch/}
ROOT~\cite{} jest środowiskiem stworzonym w CERN i Fermilab, które umożliwia pracę z wielkimi zbiorami danych, w celu ich statycznej analizy, wizualizacji i przechowywania. 
Został on stworzony w języku C++ przez co dobrze komponuje się z frameworkiem Athena i jest w nim używany na potrzeby prezentowania wyników działania algorytmów. 

\subsection{AthenaMonitoring i GenericMonitoringTool}
AthenaMonitoring jest modułem stworzonym w ramach frameworka Athena.
Dostarcza on niezbędne narzędzia do monitorowania przebiegu algorytmów i tworzenia na ich podstawie histogramów ROOT.
Proces monitoringu jest ważny dlatego, że ilość danych produkowanych przez Eksperyment ATLAS jest ogromna i jedynym sposobem, aby móc je przeanalizować są wykresy i histogramy.
Pozwalają one znaleźć odchyły od normy w danych pochodzących z eksperymentu, oraz ułatwiają obserwacje zależności pomiędzy różnymi zbiorami danych.
Podwyższa to jakość pozyskanych danych i ułatwia rozwiązywanie ewentualnych problemów.

W ramach tego modułu stworzone zostało narzędzie GenericMonitoringTool. 
Jego głównym zadaniem jest umożliwienie, w przystępny dla użytkownika sposób, dodania funkcjonalności monitoringu do algorytmów.
Realizuje to poprzez dostarczenie zbioru klas użytecznych w procesie adaptacji kodu. 
Zostało ono zaprojektowane z myślą o jak największej transparentności użycia, przy jednoczesnym zapewnieniu elastyczności konfiguracji i bezpieczeństwa przy wykorzystaniu wielowątkowym.