% !TEX root = main.tex
\section{Wstęp}

Celem niniejszej pracy jest opisanie narzędzia \mbox{GenericMonitoringTool} stworzonego w ramach frameworka Athena i opisanie jak umożliwiło one wielowątkową pracę z histogramami generowanymi przy pomocy frameworka ROOT. Histogramy są nieodłączną częścią przetwarzania i analizowania danych w instytucji takiej jak CERN, która rocznie przetwarza ok. 25 petabajtów surowych danych pochodzących z Wielkiego Zderzacza Hadronów. Histogramy pozwalają zwizualizować te dane, które w innej formie są nieczytelne dla człowieka. Umożliwiają ich analizę w poszukiwaniu schematów, prawidłowości i potwierdzenia teorii naukowych. Są także wyznacznikiem tego, czy eksperyment się udał i wyniki były zgodne z oczekiwaniami, oraz czy nie był on obarczony błędem.

\subsection{Eksperyment ATLAS}
ATLAS jest jednym z czterech głównych eksperymentów w ramach Wielkiego Zderzacza Hadronów(LHC) w ośrodku naukowym CERN. Jest to eksperyment fizyczny ogólnego przeznaczenia prowadzony przez międzynarodową grupę naukowców. Został zaprojektowany w celu pełnego wykorzystania potencjału odkrywczego i ogromnego zakresu możliwości fizycznych, jakie zapewnia LHC.

\subsection{Framework Athena}
Framework Athena jest platformą programistyczną, która umożliwia pracę z danymi pochodzącymi z eksperymentu ATLAS. Obejmuje to m.in. obsługę zdarzeń na żywo, generowanie zdarzeń, symulacje oraz rekonstrukcje. Jego głównymi komponentami są algorytmy - analizują i przekształcają dane pochodzące z eksperymentu - oraz narzędzia dla algorytmów - ułatwiają pracę z algorytmami i upraszczają ich kod. Komponenty te pisane są w języku C++ i uruchamiane z użyciem skryptów napisanych w Pythonie. 

\subsection{ROOT}
ROOT jest narzędziem stworzonym przez pracowników CERN, które umożliwia pracę z wielkimi zbiorami danych, w celu ich statycznej analizy, wizualizacji i przechowywania. Został on stworzony w języku C++ przez co dobrze komponuje się z frameworkiem Athena i jest w nim używany na potrzeby prezentowania wyników działania algorytmów. 

\comment{
\begin{itemize}
\item cel pracy -> istniejacy monitoring, ulepszenie
\item 
\item co zostalo zrobione, zeby osiagnac cel -> napisanie frameworku do monitoringu
\item dlaczego jest to wazne? -> latwiejsze uzycie kodu, utrzymanie, mniejse objectosciowo, przygotowanie do zrownoleglenia
\end{itemize}
}