% !TEX root = main.tex
\section{Podsumowanie}
GenericMonitoringTool rozwiązuje dwa największe problemy występujące przy monitorowaniu zmiennych w obrębie platformy Athena: konieczność zarządzania histogramami ROOT bezpośrednio w kodzie algorytmów oraz obsługę problemu ich synchronizacji pomiędzy wątkami.

Pierwszy z nich został rozwiązany poprzez zdefiniowanie klas reprezentujących zmienne i grupy monitorowane, oraz stworzenie centralnego modułu do zarządzania obiektami ROOT. 
Celem takiej architektury było zapewnienie przyjaznego i transparentnego dla użytkownika interfejsu do obsługi monitoringu.
Wyeliminowała ona konieczność obsługi histogramów w algorytmach, co pozwoliło usunąć z nich zbędny i powtarzalny kod.
Konsekwencją tego jest zmniejszenie liczby potencjalnych miejsc, gdzie mógł zostać popełniony błąd podczas implementacji.
Dodatkową zaletą jest to, że programiści implementujący algorytmy mogą się teraz skupić na ich prawidłowej implementacji, zamiast na obsłudze histogramów.

Rozwiązanie problemu synchronizacji wielu wątków przy wypełnianiu histogramów, nie byłoby możliwe bez refaktoringu całego rozwiązania wykonanego w ramach pierwszego problemu.
Dzięki temu, że w nowej architekturze obsługa histogramów jest scentralizowana, łatwe stało się synchronizowanie wątków próbujących modyfikować histogramy. 
Osiągnięte to zostało poprzez wprowadzenie sekcji krytycznych, osobnych dla każdego histogramu, które troszczą się o ich poprawne wypełnienie.

Przy projektowaniu i implementacji nowego rozwiązania, bardzo ważną rolę odegrała znajomość zasad tworzenia czystego kodu i refaktoringu. 
Pomogły one zidentyfikować problematyczne miejsca w kodzie, zrozumieć je i zdekomponować na małe specjalizowane klasy. 
Dzięki temu, wynikowy kod stał się bardziej czytelny, łatwiejszy do zrozumienia oraz możliwy do testowania.
Stworzenie testów pomogło wykryć problemy i błędy w dotychczasowej implementacji, oraz pozwoliło zdefiniować funkcjonalności oczekiwane od testowanego kodu. 
Wszystkie te zabiegi pozwoliły stworzyć czytelny i wysokiej jakości kod. 

Podsumowując, dzięki wszystkim powyższym zabiegom, GenericMonitoringTool stał się wartościowym następcą dawnego sposobu monitorowania algorytmów, czyniąc ten proces dużo łatwiejszym i przystępnym dla osób tworzących algorytmy.